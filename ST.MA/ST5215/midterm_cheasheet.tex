\documentclass{article}
\usepackage[a4paper, total={8in, 11.5in}]{geometry}
\usepackage{mathtools}
\usepackage{tabulary}
% commands to create lines
\newcommand{\dline}{\vspace{1pt}\hrule\hrule\vspace{1pt}}
\newcommand{\pline}{\vspace{1pt}\par\noindent\dotfill\vspace{1pt}\newline\noindent}
%override section and subsection commands
\renewcommand{\section}[1]{\\\vspace{1pt}\textbf{#1} &}
\renewcommand{\subsection}[1]{\\\vspace{1pt}\text{[#1]} &}
%remove page numbering
\pagenumbering{gobble}
% disable indent
\setlength\parindent{0pt}
% reduce padding
\renewcommand{\arraystretch}{0.5}
\setlength{\tabcolsep}{3pt}
\setlength\extrarowheight{-3pt}

\pdfinfo{
    /Title (ST5215 Advanced Statistical Theory Midterm cheat-sheet)
/Creator (Ling)}

\begin{document}

\begin{tabulary}{\textwidth}{l L}

\section{Measure and Analysis}

\subsection{Checking convexity}

Convexity of $f$ implied by positive semi-definiteness Hessian [$\mathbf{x}^T\mathbf{Ax}\geq0$]. Check (1) Non-negative eigenvalues [$\det(\mathbf{A}-\lambda\mathbf{I})= 0$], (2) Leading principal minors are positive [$\det(\triangle_k)>0$]

\subsection{Matrix operations}

$c^{T}c=c_1^2 + \cdots + c_k^2$, $cc^{T}$ is $k\times k$ matrix with $(i, j)$th element as $c_ic_j$

\subsection{$\sigma$-field}

(1) $\emptyset \in \mathcal{F}$
(2) $A \in \mathcal{F} \Rightarrow A^c \in \mathcal{F}$
(3) $A_i \in \mathcal{F} \Rightarrow \cup_{i=1}^\infty A_i \in \mathcal{F}$
(4) Smallest if any other $\sigma$-field $\subset\mathcal{E}$
(5) Borel $\mathcal{B}$ := $\sigma(\mathcal{O})$

\subsection{Measure}

(1. non-negativity) $0\leq \nu(A) \leq \infty$ $\forall$ $A \in \mathcal{F}$
(2. empty is zero) $\nu(\emptyset)=0$
(3. $\sigma$-additivity) $\sum_{i=1}^\infty \nu(A_i) \ \nu(\cup_{i=1}^\infty) A_i$ if $A_i\in\mathcal{F}$ are disjoint

\subsection{Measure property} 

(1. Monotonicity) $A\subset B \Rightarrow \nu(A) \leq \nu(B)$
(2. Sub-additivity) $\nu(\cup_{n=1}^\infty A_n) \leq \sum_{n=1}^\infty \nu(A_n)$
(3. Continuity of $\uparrow$ seq) $\lim_{n\rightarrow\infty} A_n := \cup_{n=1}^\infty A_n$ and $\nu(\lim_{n\rightarrow \infty} A_n) = \lim_{n\rightarrow \infty} \nu(A_n)$
(4. Continuity of $\downarrow$ seq) $\lim_{n\rightarrow \infty} A_n := \cap_{n=1}^\infty A_n$, and if $\nu(A_1) < \infty$ then $\nu(\lim_{n\rightarrow \infty} A_n) = \lim_{n\rightarrow \infty} \nu(A_n)$

\subsection{Measurable function}

Function $f$ is measurable function from $(\Omega, \mathcal{F})$ to $(\Lambda, \mathcal{G})$ if
$f^{-1}(A) \in \mathcal{F} \text{ for all } A \in \mathcal{G}$.
$f^{-1}(\mathcal{G})=\sigma(f)=\{f^{-1}(A):A\in\mathcal{G}\}$ is a sub-$\sigma$-field to $\mathcal{F}$

\section{Integration}

(Simple) $\int f d\nu := \sum_{i=1}^k c_i \nu(A_i)$
(Non-negative Borel) $\int f d\nu := \sup \left\{ \int f d\nu: g\in \mathcal{S}_f \right\} = \lim_n \int \varphi_n d\nu$
(Borel) $\int f d\nu := \int f_+ d\nu - \int f_- d\nu$
where $f = f_+ - f_-$, $f_+ = \max\{f(x), 0\}$, $f_- = \max\{ -f(x), 0 \}$

\subsection{Deduce $X=0$} If $X\geq 0$ a.s. and $EX=0$ then $X=0$ a.s.

\subsection{Fubini's Theorem}

Suppose $f\geq 0$ or $\int |f|d(\nu_1 \times \nu_2) < \infty$ then
$g(\omega_2) = \int_{\Omega_1} f(\omega_1, \omega_2) d\nu_1(\omega_1)$,
$\int_{\Omega_1 \times \Omega_2} f d(\nu_1 \times \nu_2) = \int_{\Omega_1}\left[ \int_{\Omega_2} f(\omega_1, \omega_2) d\nu_1(\omega_1) \right] d\nu_2(\omega_2)$

\section{Convergence}

\subsection{Monotone}

$\int \lim_n f_n d\nu = \lim_n \int f_n d\nu$
if $0\leq f_1 \leq f_2 \leq \cdots$ and $\lim_n f_n = f$ a.e.

\subsection{Fatou}

$\int \lim\inf_n f_n d\nu \leq \lim\inf_n \int f_n d\nu$
if $f_n \geq 0$

\subsection{Dominated}

$\int\lim_n f_n d\nu = \lim_n \int f_n d\nu$
if $\lim_{n\rightarrow \infty} f_n = f $ and $\exists $ integrable function $g$ s.t. $|f_n|\leq g$ a.e.

\section{Change of variables}

\subsection{Diff and Int}

$\frac{d}{d\theta}\int  f(\omega,\theta) d\nu(\omega) = \int \frac{\partial f(\omega, \theta)}{\partial\theta} d\nu(\omega)$
(1) $\exists$ $(a, b)\subset \mathcal{R}$ which $\partial f(\omega,\theta)/\partial\theta$ exists a.e. (2) $\exists$ $g$ on $\omega$ s.t. $|\partial f(\omega,\theta/\partial\theta|\leq g(\omega)$ a.e.

\subsection{Change of Var}

$\int_\Omega g \circ f d\nu = \int_\Lambda g d (\nu \circ f^{-1})$
$f$ is measurable from $(\Omega, \mathcal{F}, \nu)$ to $(\Lambda, \mathcal{G})$, 
induce measure $\nu \circ f^{-1}(B) := \nu(f^{-1}(B))\in\Lambda$ $\forall$ $B\in\mathcal{G}$.
$g$ is Borel function on $(\Lambda, \mathcal{G})$

\subsection{Change of Var Formula}

$Y=g(X)$, $A_i$ disjoint, $h_j$ is inverse function of $g$ on $A_j$.
$f_Y(y)=\sum_{j:1\leq j \leq m, y\in g(A_j)}|\det(\frac{\partial h_j(y)}{\partial y})| f_X(h_j(y))$

\subsection{CDF/Law of X/X dist} $\nu = P$, $(\Lambda, \mathcal{G}) = (\mathcal{R}, \mathcal{B})$ and $f=X$.
Then $P\circ X^{-1}$ if often denoted by $P_X$ or $F_X$ (CDF). Where $F_X(c) = P(X\leq c)$ By change of var
$E g(X) = \int_\Omega g(X(\omega)) dP(\omega) = \int_\mathcal{R} g(x) dP_X(x)  = \int_\mathcal{R} g(x) dF_X(x)$

\section{RN derivatives}

\subsection{Ab continuity}

$\lambda << \nu$ iff for any $A \in \mathcal{F}$, $\nu(A) = 0 \Rightarrow \lambda(A) = 0$

\subsection{RN PDF}

$\lambda << \nu$, there exist unique $f$ s.t. $\lambda(A) = \int_A f d\nu, A\in \mathcal{F}$.
$f=\frac{dP}{d\nu}$ is called the pdf of $P$ w.r.t. $\nu$

\section{Inequalities}

\subsection{Cauchy-Schewarz}

$Cov(X, Y)^2 \leq Var(X)Var(Y)$ \&
$(E[XY])^2 \leq EX^2 EY^2$

\subsection{Jensen}

$\varphi(EX) \leq E\varphi(X)$ if $\varphi$ is convex fn, \& 
$(EX)^{-1} < E(X^{-1})$
\& $E(logX)<log(EX)$
\& $\int f\log\left(\frac{f}{g}\right)d\nu \geq 0$

\subsection{Chebyshev}

$\varphi(t) P(|X|\geq t) \leq \int_{\{|X|\geq t\}} \varphi(X) dP \leq E\varphi(X)$, $\varphi(x)=\varphi(-x)\geq 0$, non-decreasing

\subsection{Hölders}

$E|XY| \leq (E|X|^p)^{1/p}(E|Y|^q)^{1/q}$, $1/p + 1/q = 1\Rightarrow q = p / (p-1)$

\subsection{Young}

$ab \leq \frac{a^p}{p} + \frac{b^q}{q}$
equality if and only if $a^p = b^q$

\subsection{Minkowski}

$p \geq 1$, $(E|X+Y|^p)^{1/p} \leq (E|X|^p)^{1/p} + (E|Y|^p)^{1/p}$

\subsection{Lyapunov}

$0 < s < t$, $(E|X|^s)^{1/2} \leq (E|X|^t)^{1/t}$

\end{tabulary}

\begin{tabulary}{\textwidth}{l L}

\section{CHF, MGF}

$\forall$ $t\in\mathcal{R}^d$, if $\psi$ is finite in neighborhood of $\mathbf{0}\in\mathcal{R}^d$, then moments of $X$ of any order are finite, and $\phi_X(t)=\psi_X(\sqrt{-1}t)$, $\psi_{-X}(t) = \psi_{X}(-t)$, $|\phi_X|\leq 1$, $\phi_{-X}=\overline{\phi_X(t)}$

\subsection{CHF}

$\phi_X(t) = E\left[exp(\sqrt{-1}t^TX\right]=E\left[\cos(t^TX) + \sqrt{-1}\sin(t^TX)\right]$

\subsection{MGF}

$\psi_X(t) = E\left[exp(t^TX)\right]$

\section{Conditional Exp/Dist}

\subsection{Exp}
$E(X|\mathcal{A})$ exist, unique s.t.
(1) measurable from $(\Omega, \mathcal{A})$ to $(\mathcal{R}, \mathcal{B})$
(2) $\int_C E(X|\mathcal{A})dP=\int_C X dP$, $C\in \mathcal{A}$.

[prob]
$P(B|\mathcal{A})=E(I_B|\mathcal{A})$
[Exp given $Y$]
$E(X|Y):=E[X|\sigma(Y)]$

[simple fn $Y$, discrete $A_i$] $E(X|Y) = \sum_{i=1}^\infty \frac{\int_{A_i} X dP}{P(A_i)}I_{A_i}$

\subsection{Dist}

$P_{X|Y}(B|y)$ exists on $\mathcal{B}^n\times \Lambda $ s.t.
(1) $P_{X|Y}(\cdot|y)$ is a prob. measure on $(\mathcal{R}^n, \mathcal{B}^n)$ for any fixed $y\in\Lambda$
(2) $P_{X|Y}(B|y)=P[X\in B |Y = y]$ a.s. $P_Y$ for any fixed $B\in \mathcal{B}^n$

\subsection{PDF}

(a) any fixed $y\in A$, $P_{X|Y=y}$ w.r.t $\nu$ is $f_{X|Y}(x|y) = \frac{f(x, y)}{f_Y(y)}$.
(b)If $g(x, y)$ is Borel fn on $\mathcal{R}^{n+m}$ and $Eg(X,Y) < \infty$, then
$E[g(X, Y)|Y] = \int g(x, Y)f_{X|Y}(x|Y) d\nu(x)$ a.s.

\subsection{Joint Dist}

$Q$: $\mathcal{B}^n\times \Lambda \rightarrow \mathcal{R}$ and
(1) $Q(\cdot, y)$ is a probability measure on $(\mathcal{R}^n, \mathcal{B}^n)$ for any $y\in\Lambda$
(2) $Q(B, \cdot)$ is $\mathcal{G}-$measurable for any $B\in\mathcal{B}^n$.
Then $P$ unique measure on $(\mathcal{R}^n\times\Lambda, \sigma(\mathcal{B}^n)\times\mathcal{G})$ s.t. for $B\in\mathcal{B}^n$ and $C\subset\mathcal{G}$, $P(B\times C) = \int_C Q(B, y) dP_0(y)$

\section{Exp fam}

For $\omega\in\Omega$, 
$f_{\theta}(\omega)=\frac{dP_\theta}{d\nu}(\omega) = \exp\left\{[\eta(\theta)]^T T(\omega)-\xi(\theta)\right\}h(\omega)$,
where $T$ is a random $p$-vector, $\eta$ is a function from $\Omega$ to $\mathcal{R}^p$, h is non-negative Borel function on $(\omega, \varepsilon)$ and 
$\xi(\theta) =\log\left\{\int_\Omega \exp\left\{[\eta(\theta)]^T T(\omega)\right\}h(\omega) d\nu(\omega)\right\}$.
Joint exp is exp.

\subsection{NEF}

$f_\eta(\omega) = \exp\left\{\eta^T T(\omega)-\mathcal{C}(\eta)\right\}h(\omega)$
$\mathcal{C}(\eta)=\log\left\{\int_\Omega \exp\left\{\eta^T T(\omega)\right\}h(\omega)d\nu(\omega)\right\}$
$\eta$ is called natural parameter and natural parameter space
$\Xi=\{\eta(\theta):\theta\in\Theta\}\subset\mathcal{R}^p$. Full rank if $\Xi$ contains open set in $\mathcal{R}^p$

\subsection{Show non Exp Fam} For an exp fam $P_\theta$, there is nonzero measure $\lambda$ s.t. $\frac{dP_\theta}{d\lambda}(\omega)>0$ $\lambda$-a.e. and for all $\theta$.

Consider $f=\frac{dP_\theta}{d\lambda}I_{(t, \infty)}(x)$,
$\int fd\lambda=0, f\geq 0\Rightarrow f=0$. Since $\frac{dP_\theta}{d\lambda}>0$ (assume), then $I_{(t, \infty)}(x)=0\Rightarrow v([t, \infty))=0$. Since $t$ is arbitary, consider $v(\mathcal{R})=0$ (contradiction)

\subsection{Split $T(X)$} Let $T=(Y, U)$ and $\eta=(\nu, \varphi)$ [$Y$, $\nu$ same dim]. Then $Y$ has PDF
$f_\eta(y)=\exp\left\{\nu^T y - \mathcal{C}(\eta)\right\}$

w.r.t $\sigma$-finite measure depending on $\varphi$.
If $T$ has a PDF in NEF, the conditional distirbution of $Y$ given $U=u$ has PDF
$f_{\nu, u}(y) = \exp\left\{\nu^T y - \mathcal{C}_u(\nu)\right\}$
which is in a NEF indexed by $\nu$

\subsection{MGF of NEF}

If $\eta_0$ is an interior point on natural parameter space, then MGF $\phi_{\eta_0}(t)$ of $T$ (with $P=P_{\eta_0}$ is finite in neighborhood of $t=0$ and is given by
$\psi_{\eta_0}(t) = \exp\left\{\mathcal{C}(\eta_0+t)-\mathcal{C}(\eta_0)\right\}$

Let $A(\theta)=\mathcal{C}(\eta_0(\theta))$, $\frac{dA(\theta)}{d\theta}=\frac{d\mathcal{C}(\eta_0(\theta))}{d\eta_0(\theta)}\cdot\frac{d\eta_0(\theta)}{d\theta}$,
$E_{\eta_0} T = \frac{d\psi_{\eta_0}}{dt}|_{t=0} = \frac{d\mathcal{C}}{d\eta_0}
=\frac{A'(\theta)}{\eta_0'(\theta)}$,
$E_{\eta_0}T^2 = \mathcal{C}''(\eta_0) + \mathcal{C}'(\eta_0)^2$,
$Var(T) = \mathcal{C}''(\eta_0) = \frac{A''(\theta)}{[\eta_0(\theta)]^2} - \frac{\eta_0(\theta)''A'(\theta)}{[\eta_0(\theta)']^3}$

\subsection{Differential identities}

$G(\eta) := \int g(\omega) \exp\left\{\eta^T T(\omega)-\mathcal{C}(\eta)\right\}h(\omega) d\nu(\omega)$,
For $\eta$ in interior of $\Xi_g$,
(1) $G$ is continuous and has continuous derivatives of all orders.
(2) These derivatives can be computed by differentiation under the integral sign.
e.g. $\frac{dG(\eta)}{d\eta} = E_\eta \left[g(\omega) \left(T(\omega) - \frac{\partial}{\partial\eta}\xi(\eta)\right)\right]$

\section{Statistics}

[Sample Variance] $S^2=\frac{1}{n-1}\sum_i(X_i-\bar{X})^2$
[Ordered stats] $X_{(k)}$, $X_{(n)}=[F(x)]^n, f_{X_{(n)}}=nf(x)[F(x)]^{n-1}$, $X_{(1)}=1-[1-F(x)]^n, f_{X_{(1)}}=nf(x)[1-F(x)]^{n-1}$

\section{Sufficiency}

$P(X=x|T=t) \text{ does not depend on } \theta$

\subsection{Factorization}

$\frac{dP}{d\nu}(x) = g_P(T(x))h(x)$, 
(1) $h(x)$ does not depend on $P$
(2) $g_P(t)$ depends on $P$

\section{Minimal sufficiency}

Let $T$ be a sufficient statistics for $P\in\mathcal{P}$. $T$ is called minimal sufficient statistics iff for any other statistics $S$ sufficient for $P\in\mathcal{P}$, there is a measurable function $\psi$ s.t. $T=\psi(S)$ $\mathcal{P}$-a.s.

\subsection{Method1 (A+B)}

[Theorem A]
Suppose $\mathcal{P}_0\subset\mathcal{P}$ and $\mathcal{P}_0$-a.s. implies $\mathcal{P}$-a.s.
If $T$ is sufficient for $P\in\mathcal{P}$ and minimal sufficient for $P\in\mathcal{P}_0$, then $T$ is minimal sufficient for $P\in\mathcal{P}$

[Theorem B]
Suppose $\mathcal{P}$ contains PDFs $f_0, f_1, \cdots$ w.r.t a $\sigma$-finite measure.

(1) Define $f_\infty(x)=\sum_{i=0}^\infty c_if_i(x)$, $T_i(x)=f_i(x)/f_\infty(x)$, then $T(X)=(T_0(X), T_1(X), \cdots)$ is minimal sufficient for $\mathcal{P}$. Where $c_i>0, \sum_{i=0}^\infty c_i=1, f_{\infty}(x)>0$.

(2) If $\{x:f_i(x)>0\}\subset \{x: f_0(x) > 0\}$ for all $i$, then $T(X)=(f_1(x)/f_0(x), f_2(x)/f_0(x), \cdots$ is minimal sufficient for $\mathcal{P}$

\subsection{Method2 (C)}

[Theorem C]
Suppose $\mathcal{P}$ contains PDFs $f_P$ w.r.t. $\sigma$-finite measure $\nu$. If
(a) $T(X)$ is a sufficient statistics, and
(b) There is a measurable function $\phi$ s.t. for any possible values $x, y$ of $X$, or $x, y\in\{x:h(x)>0\}$ for NEF.

$f_P(x) = f_P(y)\phi(x, y) \forall P\in\mathcal{P} \Rightarrow T(x)=T(y)$,
then $T(X)$ is minimal sufficient for $\mathcal{P}$

\end{tabulary}

\begin{tabulary}{\textwidth}{l L}

\subsection{NEF special}

If there exists $\Theta_0=\{\theta_0, \theta_1, \cdots, \theta_p\}\subset \Theta$ s.t. vectors $\eta_i=\eta(\theta_i)-\eta(\theta_0), i \in [1, p]$ are linearly independent in $\mathcal{R}^p$, then $T$ is also minimal sufficient.
Check $det([\eta_1, \cdots, \eta_p])$ is non-zero OR $\Xi = \{\eta(\theta):\theta\in\Theta\}$  contains $(p+1)$ points that do not lie on the same hyperplane OR $\Xi$ is full rank.

\section{Completeness}

Statistics $T(X)$ is complete for $P\in\mathcal{P}$ iff for any Borel function $f$, $E_P f(T)=0$ for all $P\in\mathcal{P}$ implies $f(T)=0$ $\mathcal{P}$-a.s.
$T$ is boundedly complete iff statements holds for bounded Borel functions $f$.

[Completeness + Sufficiency $\Rightarrow$ Minimal Sufficiency]
Suppose $X$ is a sample from unknown $P\in\mathcal{P}$, and suppose a minmal sufficient statistics exists.
If a statistics $U$ is sufficient and boundedly complete, then $U$ is minimal sufficient

\subsection{Ancillary statistics}
A statistics $V(X)$ is ancillary for $\mathcal{P}$ if its distribution does not depend on population $P\in\mathcal{P}$

[First-order ancillary] if $E_P[V(X)]$ does not depend on $P\in\mathcal{P}$

\subsection{NEF special}

If $\mathcal{P}$ is NEF of full rank then $T(X)$ is complete and sufficient for $\eta\in\Xi$

\section{Basu's theorem}

Let $V$ and $T$ be two statistics of $X$ from a population $P\in\mathcal{P}$. If $V$ is ancillary and $T$ is boundedly complete and sufficient for $P\in\mathcal{P}$, then $V$ and $T$ are independent w.r.t any $P\in\mathcal{P}$ 

\section{Point Estimate}

\subsection{Method of Moments}
[$j$th moments of $P_\theta$] $\mu_j = E_\theta X^j$

[Unbiased estimate] $\hat\mu_j = \frac{1}{n}\sum_{i=1}^n X^j$

[MoM] Find the first $n$ moments, $n=$ number of parameters. Note $EX^2=Var(X)+(EX)^2$

\subsection{Maximum Likelihood}

[Likelihood function] $\ell(\theta)=f_{\theta}(x)$

[Maximum likelihood estimate of $\theta$] $\hat\theta \in \Theta$ satisfying $\ell(\hat\theta)=\max_{\theta\in\Theta}\ell(\theta)$

[Maximum likelihood estimator] If $\hat\theta$ is a Borel function $X$ a.e. $\nu$

[MLE est] $\max_{\theta}\ell(\theta)$

[Limitations] MLE might not exist, solvable, no PDFs, not MSE optimal. Note need to check all critical points

\end{tabulary}

    \section{Distributions (Prior and Predictive)}
    \begin{align*}
        p(\theta)
    \end{align*}

    \subsection{Beta distribution}

    $a > 0, b > 0, B(a, b) = \int_0^1 x^{a-1}(1-x)^{b-1}dx$
    \begin{align*}
        p(\theta) & \propto \theta^{a-1}(1-\theta)^{b-1}
        , \theta \in (0, 1)\\
                  &= \frac{1}{B(a, b)}\theta^{a-1}(1-\theta)^{b-1}
    \end{align*}

    \subsubsection{Properties}
    \begin{align*}
        E(\theta) &= \frac{a}{a+b}\\
        Var(\theta) &= \frac{ab}{(a+b)^2(a+b+1)}\\
        Mode(\theta) &= \frac{a-1}{a+b-2}, \text{ if } a > 1, b>1\\
        B(a, b) &= \frac{\Gamma(a)\Gamma(b)}{\Gamma(a+b)}\\
        \Gamma(a) &= \int_0^\infty z^{a-1}e^{-z}dz
    \end{align*}

    \subsubsection{Choice of $a$ and $b$}

    Given the mean (or proportion), we set
    $E(\theta)=\frac{a}{a+b}=mean$

    Choosing large $a, b$ result in smaller variance (avoid)

    \subsection{Gamma distribution}

    $\theta \sim Gamma(a, b), a > 0, b > 0$

    \begin{align*}
        p(\theta) &=
        \frac{b^a}{\Gamma(a)}\theta^{a-1}e^{-b\theta},
        \theta > 0
    \end{align*}

    $\Gamma(a)=\int_0^\infty x^{a-1}e^{-x}dx=(a-1)!$


    \subsubsection{Properties}

    \begin{itemize}
        \item $E(\theta) = \frac{a}{b}$\\
        \item $Var(\theta) = \frac{a}{b^2}$\\
        \item $Mode(\theta) = \frac{a-1}{b}, a > 1$
        \item $cX \sim \Gamma(a, \frac{b}{c})$
        \item Chi-square: $\chi_{\nu}^2 \sim
            \Gamma(\frac{\nu}{2}, \frac{1}{2})$
    \end{itemize}

    If $a=1 \Rightarrow p(\theta)\sim exp(b)$ with mean
    $\frac{1}{b}$

    \subsubsection{Selection of $a$ and $b$}

    With given mean, a large $b$ will result in smaller
    variance ($Var(\theta)=\frac{E(\theta)}{b}$)

    \subsection{Negative binomial distribution}

    $Y\sim NB(r, p)$ if
    \begin{align*}
        P(Y=y) &= \begin{pmatrix}
            r + y - 1\\
            y
            \end{pmatrix} p^r (1-p)^y, y = 0, 1, 2, \cdots
    \end{align*}

    This is the probability of $r$ success and $y$ number of
    failures (note we count the failures)

    \subsection{Dirichlet distribution}

    $Dir(a_1, \cdots, a_K)$ is extension of $Beta$ prior to $K \geq 2$

    \begin{align*}
        p(\theta) &= \frac{1}{B(a_1, \cdots,
            a_K)}\theta_1^{a_1-1}\cdots \theta_K^{a_K-1}\\
            a_i \leq 1,~ & \sum_{i\in K}\theta_i = 1
    \end{align*}

    where $B(a_1, \cdots, a_K) =
    \frac{\Gamma(a_1)\times\cdots\Gamma(a_K)}{\Gamma(a_1+\cdots+a_K)}$

    \subsubsection{Marginal Dirichlet is Beta}

    \begin{align*}
        p(\theta_i | \theta^{(-i)}) \sim Beta\left(a_i,
        \sum_{j\neq i} a_j\right)
    \end{align*}

    \subsection{Normal distribution}

    $Y\sim(\theta, \sigma^2)$
    \begin{align*}
        p(y|\theta, \sigma^2) &=
        \frac{1}{\sqrt{2\pi\sigma^2}}exp\left(-\frac{(y-\theta)^2}{2\sigma^2}\right)
    \end{align*}

    Properties
    \begin{itemize}
        \item Symmetric about $\theta$. Median and mode are
            both $\theta$
        \item Affine transformation of Normal is Normal
        \item Standard normal $Z = \frac{Y-\theta}{\sigma}$
        \item Precision of sum is sum of precision $W = \sum_{i\in N} \lambda_i$
        \item $Q/(n-1)\sim \Gamma(\frac{n-1}{2},
            \frac{n-1}{2})$
            \subitem $Y_1, \cdots, Y_n \sim N(\theta,
            \sigma^2)$ independently
            \subitem $Q:= \frac{1}{\sigma^2}\sum_{i=1}^n (Y_i -
            \bar{Y})^2 \sim \chi_{n-1}^2$
    \end{itemize}

    \subsubsection{Precision}

    Precise $\lambda$: measurements tend to be close to each other
    \begin{align*}
        \lambda = \frac{1}{\sigma^2}
                =\Sigma^{-1}
    \end{align*}

    \subsubsection{Highest precision (smallest variance)}

    Let $Y_1\sim N(\theta, \sigma_1^2), Y_2 \sim N(\theta,
    \sigma_2^2)$
    \begin{align*}
        Z &= wY_1 + (1-w)Y_2\\
        \Rightarrow E(Z) &= E(wY_1 + (1-w)Y_2) = \theta\\
        \Rightarrow Var(Z) &= w^2 Var(Y_1) + (1-w)^2
        Var(Y_2)\\
        \text{solve } \min_w & \frac{w^2}{\lambda_1} +
        \frac{(1-w)^2}{\lambda_2}\\
        \Rightarrow w^* &=
        \frac{\lambda_1}{\lambda1+\lambda2}\\
        \Rightarrow Var_{\min}(Z) &=
        \left(\frac{\lambda_1}{\lambda_1+\lambda_2}\right)^2
        \frac{1}{\lambda_1}
        + 
        \left(\frac{\lambda_2}{\lambda_1+\lambda_2}\right)^2
        \frac{1}{\lambda_2}\\
                           &= \frac{1}{\lambda_1+\lambda_2}
    \end{align*}

    Generally the precision of $Z$ is the sum of precision
    of $Y$s
    \begin{align*}
        Z &= \left(\frac{\lambda_1}{\sum_{i\in
        n}\lambda_i}\right) Y_1 + \cdots +
        \left(\frac{\lambda_n}{\sum_{i\in
        n}\lambda_i}\right) Y_n\\
          &= \frac{1}{\sum_{i\in n}\lambda_i}\\
                    \Rightarrow \lambda_Z &= {\sum_{i\in
                    n}\lambda_i}\\
    \end{align*}

    \subsection{Student's t-distribution}

    $Z\sim N(0, 1), Q\sim \chi_\nu^2$
    \begin{align*}
        T = \frac{z}{\sqrt{Q/\nu}}\sim t_\nu
    \end{align*}

    Properties
    \begin{itemize}
        \item $T = \frac{\sqrt{n}(\bar{Y}-\theta)}{s_Y}\sim
            t_{n-1}$
            \subitem $Y_1, \cdots, Y_n \sim N(\theta,
            \sigma^2)$ independently
            \subitem $\bar{Y} = \frac{1}{n}\sum_i Y_i$
            \subitem $S_Y =
            \sqrt{\frac{1}{n-1}\sum_i(Y_i-\bar{Y})^2}$
    \end{itemize}


\end{document}