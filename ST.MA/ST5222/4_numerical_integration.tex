\section{L3: Numerical Integration}

\begin{tabulary}{\textwidth}{l L}

\subsection{Integration}

Objective: approximate $\int_a^b f(x) dx$ numerically

Naive method: Divide $[a, b]$ into $n$ sub-intervals,  $x^*_i$ is the middle point of $i$th subinterval.

$$\int_a^b f(x) dx \approx \frac{b-a}{n}\sum_{i=1}^n f(x_i^*)$$

Improvement: for each of the sub-interval $[x_i, x_{i+1}]$ add $(m+1)$ nodes

\subsection{Trapezoidal Rule}

Choose 2 nodes ($m=1$) in $[x_i, x_{i+1}]$.
To approximate height $I = \frac{\int_{x_i}^{x_{i+1}} f(x) dx }{x_{i+1} -x_i}$. Area = $(x_{i+1} - x_{i})\times I$

$$
\hat I_1 = \frac{f(x^*_0) + f(x^*_1)}{2}
$$

Total area $\int_a^b f(x) dx$, with $h = (b-a)/n$

$$
\hat T(n) = h \sum_{i=1}^n \frac{f(x_i) + f(x_{i+1})}{2}
$$

$\hat T(n) - \int_a^b f(x)dx = O(n^{-2})$

\subsection{Simpson Rule}

Choose 3 nodes ($m=2$). Approximate height $I$

$$
\hat I_2 = \frac{1}{6} f(X^*_0) + \frac{4}{6} f(x_1^*) + \frac{1}{6} f(x^*_2)
$$

Total area $\int_a^b f(x) dx$, with $h = (b-a)/n$, $x^*_i = (x_i+x_{i+1})/2$

$$
\hat S(n) = h\sum_{i=1}^n \left\{
\frac{f(x_i)}{6} + \frac{2f(x_i^*)}{3} + \frac{f(x_{i+1})}{6}
\right\}
$$

$\hat S(n) - \int_a^b f(x) dx = O(n^{-4})$, can generalised to other polynomial order $m$

\subsection{Gaussian Quadrature}

Perfect est for polynomial order $2m+1$ and below (or fn close enough) using $2m+2$ points.

$$
I = \int_a^b w(x) f(x) dx = \sum_{j=0}^m c_j f(x_j)
$$

when $a, b$ finite, $w(x)=1$; when $a=0, b=\infty$, $w(x) = e^{-x}$; when $a=-\infty, b=\infty$, $w(x) = e^{-x^2/2}$

Requires solving for $x_0, \cdots, x_m$ and $c_0, \cdots, c_m$ ($2m+2$ unknowns)

\end{tabulary}