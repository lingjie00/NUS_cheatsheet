\documentclass[a4paper,12pt,landscape]{article}
\usepackage{multicol}
\usepackage{calc}
\usepackage{ifthen}
\usepackage[a4paper, landscape]{geometry}
\usepackage{amsmath,amsthm,amsfonts,amssymb}
\usepackage{color,graphicx,overpic}
\usepackage{hyperref}
\usepackage{tabulary}
\usepackage{soul} %for highlight
\usepackage{xcolor} %color definition
\usepackage{sectsty} %change section color
\usepackage{tabulary} % better table
\usepackage{makecell} % new line in cell

% type codes
\usepackage{listings} 
\usepackage{xcolor}

\definecolor{codegreen}{rgb}{0,0.6,0}
\definecolor{codegray}{rgb}{0.5,0.5,0.5}
\definecolor{codepurple}{rgb}{0.58,0,0.82}
\definecolor{backcolour}{rgb}{0.95,0.95,0.92}

\lstdefinestyle{mystyle}{
    backgroundcolor=\color{backcolour},   
    commentstyle=\color{codegreen},
    keywordstyle=\color{magenta},
    numberstyle=\tiny\color{codegray},
    stringstyle=\color{codepurple},
    basicstyle=\ttfamily\footnotesize,
    breakatwhitespace=false,         
    breaklines=false,                 
    captionpos=b,                    
    keepspaces=false,                 
    numbers=left,                    
    numbersep=5pt,                  
    showspaces=false,                
    showstringspaces=false,
    showtabs=false,                  
    tabsize=2
}

\lstset{style=mystyle}

\pdfinfo{
  /Title (EC4305 Applied Econometrics)
  /Creator (Ling)}

% This sets page margins to .5 inch if using letter paper, and to 1cm
% if using A4 paper. (This probably isn't strictly necessary.)
% If using another size paper, use default 1cm margins.
\ifthenelse{\lengthtest { \paperwidth = 11in}}
    { \geometry{top=.5in,left=.5in,right=.5in,bottom=.5in} }
    {\ifthenelse{ \lengthtest{ \paperwidth = 297mm}}
        {\geometry{top=.5cm,left=.5cm,right=.5cm,bottom=.5cm} }
        {\geometry{top=.5cm,left=.5cm,right=.5cm,bottom=.5cm} }
    }

% Turn off header and footer
\pagestyle{empty}

% Redefine section commands to use less space
\makeatletter
\renewcommand{\section}{\@startsection{section}{1}{0mm}%
                                {-1ex plus -.5ex minus -.2ex}%
                                {0.5ex plus .2ex}%x
                                {\normalfont\large\bfseries\color{red}}}
\renewcommand{\subsection}{\@startsection{subsection}{2}{0mm}%
                                {-1explus -.5ex minus -.2ex}%
                                {0.5ex plus .2ex}%
                                {\normalfont\normalsize\bfseries\color{blue}}}
\renewcommand{\subsubsection}{\@startsection{subsubsection}{3}{0mm}%
                                {-1ex plus -.5ex minus -.2ex}%
                                {1ex plus .2ex}%
                                {\normalfont\small\bfseries\color{violet}}}
\makeatother

% Define BibTeX command
\def\BibTeX{{\rm B\kern-.05em{\sc i\kern-.025em b}\kern-.08em
    T\kern-.1667em\lower.7ex\hbox{E}\kern-.125emX}}

% Don't print section numbers
\setcounter{secnumdepth}{0}


\setlength{\parindent}{0pt}
\setlength{\parskip}{0pt plus 0.5ex}

%My Environments
\newtheorem{example}[section]{Example}
% -----------------------------------------------------------------------

\begin{document}
\raggedright
\footnotesize
\begin{multicols}{3}


% multicol parameters
% These lengths are set only within the two main columns
\setlength{\columnseprule}{0.25pt}
\setlength{\premulticols}{1pt}
\setlength{\postmulticols}{1pt}
\setlength{\multicolsep}{1pt}
\setlength{\columnsep}{2pt}

\begin{center}
     \Large{\underline{EC4305 Applied Econometrics}} \\
     {Lingjie, \today}
\end{center}

Objective: understanding causal relationships between variables.

Require: dataset with external shock to determine the causal impact

Example: Do hospitals make people healthier?

\section{Acronyms}

\begin{tabulary}{\linewidth}{l @{ : } L}
    $Y_i$ & outcome for individual $i$\\
    $D_i$ & treatment of individual $i$\\
    $Y_{1i}$ & outcome for individual $i$ when given treatment\\
    $Y_{0i}$ & outcome for individual $i$ when not given treatment\\
    ATE & $E\left[ Y_{1i} - Y_{0i}  \right]$\\ & Average Treatment Effects\\
    TOT & $E\left[ Y_{1i} - Y_{0i} | D_i = 1 \right]$ \\
        & Treatment effect Of the Treated\\
    ITT & $E(Y|\text{ compliers \& non-compliers}) - E(Y|\text{ untreated})$ \\
        & Intent to Treat\\
    counterfactual & $E\left[ Y_{0i} | D_i=1  \right]$\\
                   & (unobserved) outcome for individual in an alternative universe without treatment \\
    selection bias & $E\left[ Y_{0i}|D_i=1  \right] - E\left[ Y_{0i}|D_i=0  \right]$ \\
                   & bias due to omitted variable\\
    observed difference & $E\left[ Y_{1i}|D_i=1  \right] - E\left[ Y_{0i}|D_i=0  \right]$\\
                        & observed difference in average outcome
\end{tabulary}

\section{Causal Inference}

Outcome:
\begin{align*}
    Y_i &= Y_{0i} + (Y_{1i} - Y_{0i})D_i, ~D_i\in\{0, 1\}
\end{align*}

Where $Y_{1i} - Y_{0i}$ is the causal effect for individual $i$

\subsection{Regression results}

\subsubsection{RCT}

\begin{align*}
    Y_i &= \alpha + \beta\cdot D_i + \epsilon_i
\end{align*}
$\alpha = E\left[ Y_i|D_i=0  \right]$\\
$\beta = E\left[ Y_i|D_i=1  \right] - E\left[ Y_i|D_i=0  \right]$

\begin{lstlisting}
reg Y X
\end{lstlisting}

\subsubsection{FE}

Note: $\lambda_t = \sum_m \rho_m \cdot I_m$

\begin{align*}
    Y_{it} = \lambda_t
\end{align*}

$\lambda_t = E[Y_{it}|t]$ 

\begin{align*}
    Y_{it} = \alpha + \lambda_t
\end{align*}
$\alpha = E[Y_{it}|t=\text{base}]$\\
$\lambda_t = E[Y_{it}|t] - E[Y_{it}|t=\text{base}]$

\subsubsection{Dynamic Effect}

Note: $\lambda_t \cdot D_i = \sum_m \rho_m \cdot D_i \cdot I_m$\\
(causal effect is same with or without constant) $\lambda_t \cdot D_i = E[Y_{it}|t, D_i=1]-E[Y_{it}|t, D_i=0]$

\begin{align*}
    Y_{it} &= \lambda_t + \lambda_t \cdot D_i\\
\end{align*}
$\lambda_t = E[Y_{it}|D_i=0]$\\

\begin{align*}
    Y_{it} &= \alpha + \lambda_t + \lambda_t \cdot D_i\\
\end{align*}
$\alpha = E[Y_{it}|t=\text{base}, D_i=0]$\\
$\lambda_t = E[Y_{it}|D_i=0]-E[Y_{it}|t=\text{base}, D_i=0]$\\

\subsubsection{Time Trend}

\begin{align*}
    Y_{it} &= \alpha + \beta Post_t
\end{align*}
$\alpha = E[Y_{it}|Post_t=0]$\\
$\beta = E[Y_{it}|Post_t=1]-E[Y_{it}|Post_t=0]$

\begin{align*}
    Y_{it} &= \alpha + \beta D_i + \gamma Post_t
\end{align*}
$\alpha = E[Y_{it}|D_i=0, Post_t=0]$\\
$\gamma = E[Y_{it}|D_i=0, Post_t=1]-E[Y_{it}|D_i=0, Post_t=0]$
$\beta = E[Y_{it}-\gamma\cdot Post_t|D_i=1]-E[Y_{it}-\gamma\cdot Post_t|D_i=0]$\\
Note: $\beta$ relates to $-\gamma\cdot Post_t$ to control for time trend.
Control the fact that everyone is healthier/sicker in the post-period due to macroeconomics conditions.

\subsubsection{DID}

\begin{align*}
    Y_{it} &= \alpha + \beta D_i + \gamma Post_t + \delta D_i \cdot Post_t
\end{align*}

$\alpha = E[Y_{i0}|D_i=0]$\\
$\beta = E[Y_{i0}|D_i=1]-E[Y_{i0}|D_i=0]$\\
$\gamma = E[Y_{i1}|D_i=0] - E[Y_{i0}|D_i=0]$\\
$\delta = \left(E[Y_{i1}|D_i=1]-E[Y_{i0}|D_i=1]\right) - \left( E[Y_{i1}|D_i=0] - E[Y_{i0}|D_i=0] \right)$ (DID
estimator)

Note:
\begin{itemize}
    \item always Post minus Pre (base), Treat minus Control (base)
    \item $E[Y_{it}|Post_t=0] = E[Y_{i0}]$
    \item $E[Y_{it}|Post_t=1] = E[Y_{i1}]$
\end{itemize}

\subsubsection{DID with FE}

\begin{align*}
    Y_{it} &= \alpha + \beta D_i + \lambda_t + \delta D_i \cdot Post_t
\end{align*}

Note: $Post_t = \sum_m \lambda_t$ (post treatment time dummies)\\
Useful for granular control of time trend and smaller standard errors.

\subsubsection{DID with two-way FE}

\begin{align*}
    Y_{it} &= \alpha + \beta D_i + \lambda_t + \delta D_i \cdot Post_t
\end{align*}

Note: $D_i = \sum_j \alpha_i$ (entity dummies)\\
Useful for granular control of time trend and smaller standard errors.

\subsection{Selection bias}

Observed difference = TOT + selection bias

\begin{align*}
    &E\left[ Y_i|D_i=1  \right] - E\left[ Y_i|D_i=0  \right]\\
    &=E\left[ Y_{1i}|D_i=1  \right] - E\left[ Y_{0i}|D_i=0  \right]\\
    &=E\left[ Y_{1i}|D_i=1  \right] - E\left[ Y_{0i} | D_i = 1  \right]\\
    &+E\left[ Y_{0i}|D_i=1  \right] - E\left[ Y_{0i} | D_i = 0  \right]
\end{align*}

Selection bias = difference in outcome between
treatment and control group, before any treatment\\
e.g. health status of older vs younger people before going to hospital

\subsubsection{Key questions}

\begin{align*}
    E\left[ Y_{0i}|D_i=1  \right] - E\left[ Y_{0i} | D_i = 0  \right]
\end{align*}

\begin{itemize}
    \item Direction
    \item Magnitude
    \item r/s between outcome and treatment selection
        \begin{align*}
            Cov(Y_{0i}, D_i)
        \end{align*}
        \subitem who is more likely to be selected for treatment
    \item Over-estimation of causal effect: $b > a$
    \item Under-estimation of causal effect: $b < a$
\end{itemize}

$a$ := actual treatment effect\\
$b$ := estimated treatment effect (might be biased)

\subsection{ITT vs TOT}

\begin{tabulary}{\linewidth}{l @{ : } L}
    $A$ & offered program, treated\\
    $B$ & offered program, not treated\\
    $C$ & not offered program
\end{tabulary}

\begin{tabulary}{\linewidth}{l | L | L}
    \hline
    & ITT & TOT \\
    \hline
    \hline
    name & Intent to Treat & Treatment on the Treated\\
    \hline
    measures & effect of making eligible for treatment & effect of taking treatment\\
    \hline
    causal effect & yes & no, unless 100\% compliers\\
    \hline
    calculation & $E(Y|A~\&B)-E(Y|C)$ & $E(Y|A) - E(Y|C)$\\
    \hline
\end{tabulary}

\begin{align*}
    y_{it} = \alpha + \beta_1 (\text{PostOffer}_{it}) + \beta_2 (\text{PreOffer}_{it}) + X_i\Gamma + \gamma_t + \epsilon_{it}
\end{align*}

\begin{tabulary}{\linewidth}{l @{ : } L}
    $\beta_1$ & ITT\\
    $\beta_2$ & waiting list effect\\
    $\Gamma$ & control for other covariants\\
    $\gamma_t$ & time dummy
\end{tabulary}

\subsection{Dynamic effect}

Effect of a treatment might change over time.
We might be interested in the dynamics of the ATE.

Creating $N-1$ time dummies
\begin{align*}
    &I(Jul=1) + I(Aug=1) + I(Sep=1) = 1 \\
    &Y_i = \alpha + \beta_{Aug} I_{Aug} + \beta_{Sep} I_{Sep} + \epsilon_i
\end{align*}

Coefficient of $I_{Aug}$ measures the average outcome of Aug to the base group (Jul).

\subsubsection{Dynamic Effect of a treatment}

\begin{align*}
    Y_i = & \alpha + \sum_{m=0}^3\rho_m D_i I_m + \sum_{m=1}^3 \beta_m I_m + \epsilon_i\\
    \Leftrightarrow & \alpha + \sum_{m=0}^3 \rho_m D_i I_m + \gamma_t + \epsilon_i
\end{align*}

\begin{tabulary}{\linewidth}{l @{ : } L}
    $\rho_m$ & dynamic effect, average treatment effect in month $m$\\
    $I_m$ & indicator function for month $m$\\
    $D_i$ & treatment of entity $i$\\
    $\gamma_t$ & monthly fixed effect
\end{tabulary}

Note: we can include all interactions but not all month dummies.

\section{Randomization}

Randomization ensures no selection bias

\begin{align*}
    \{ Y_{0i}, Y_{1i}  \} \perp D_i
\end{align*}
\begin{align*}
    &\Rightarrow E[Y_{0i}|D_i=1] = E[Y_{0i}|D_i=0] = E[Y_{0i}]\\
    &\Rightarrow E[Y_{1i}|D_i=1] = E[Y_{1i}|D_i=0] = E[Y_{1i}]
\end{align*}

Therefore 
\begin{align*}
    &E\left[ Y_{0i}|D_i=1  \right] - E\left[ Y_{0i}|D_i=0  \right] = 0\\
    &\Rightarrow E[Y_{1i} | D_i = 1] - E[Y_{0i}|D_i=0] \\
    &= E[Y_{1i}-Y_{0i}|D_i=1]\\
    &=E[Y_{1i}-Y_{0i}]
\end{align*}

TOT = ATE under randomization

\subsubsection{Limitation}

\begin{enumerate}
    \item Budget, ethical constraints
    \item Impossible to conduct experiment
    \item Analysing past data/ pilot program before experiment
\end{enumerate}

\subsection{Regression Analysis of Experiments}

Regression model
\begin{align*}
    Y_i = \beta_0 + \beta_1 D_i + \epsilon_i
\end{align*}

Interpretation

\begin{tabulary}{\linewidth}{l @{ = } l l}
    $\beta_0$ & $E[Y_{0i}|D_i=0]$ & (base effect)\\
    $\beta_1$ & $E[Y_i|D_i=1] - E[Y_i|D_i=0]$ & (observed difference)\\
              & $E[Y_{1i} - Y_{0i} |D_i=1]$ & (TOT)\\
              & $E[Y_{1i} - Y_{0i}]$ & (ATE)\\
\end{tabulary}

\subsection{Balanced test}

\begin{tabular}{l @{ : } l}
    Objective & Check $E[Y_{0i}|D_i=1] - E[Y_{0i}|D_i=0] =0$\\
    Limitation & $E[Y_{0i}|D_i=1]$ is unobserved\\
    Solution & check pre-treated outcomes\\
    Ideal & pre-treated outcomes are all balanced
\end{tabular}

Check
\begin{align*}
    E[X_i|D_i=1] = E[X_i|D_i=0]
\end{align*}
Note: This is sufficient but not necessary condition. i.e.\\
No selection bias $\Rightarrow$ balanced pre-treated outcomes

\begin{lstlisting}
ttest X, by(treatment)
\end{lstlisting}

\subsection{Conditional Independence Assumption}

Random assignment conditional on a group\\
$\Rightarrow$ selection bias is zero within the group

\begin{align*}
    \{ Y_{0i}, Y_{1i}  \} \perp D_i | F_i
\end{align*}
\begin{align*}
    &\Rightarrow E[Y_{0i}|D_i=1] \neq E[Y_{0i}|D_i=0]\\
    &\Rightarrow E[Y_{0i}|D_i=1,F_i=1] = E[Y_{0i}|D_i=0,F_i=1]
\end{align*}

\subsubsection{Regression analysis}

Assume CIA
\begin{tabular}{l @{ : } l}
    Short regression & $ Y_i = \alpha_0 + \alpha_1 D_i + u_i$\\
                     & $\alpha_1=$ TOT + SB\\
    Long regression & $ Y_i = \beta_0 + \beta_1 D_i + \beta_2 F_i + \epsilon_i$\\
                    & $\beta_1=$ TOT
\end{tabular}

\subsubsection{Adding irrelevant regressors}

\begin{tabulary}{\linewidth}{l @{ : } L}
    unconditional & adding any variables (including $F$) changes $\beta$ little\\
    CIA & $\beta$ sensitive to inclusion of $F$ (should include)\\
        & $\beta$ changes little when adding other variables (excluding $F$)
\end{tabulary}

\begin{align*}
    &(1) ~Y_i = \rho_1 D_i + u_i\\
    &(2) ~Y_i = \rho_2 D_i + \delta F_i + u_i\\
    &(3) ~Y_i = \rho_3 D_i + \gamma C_i + u_i\\
    &(4) ~Y_i = \rho_4 D_i + \delta_2 F_i + \gamma_2 C_i + u_i
\end{align*}

$\rho_1 \approx \rho_3 \neq \rho_2 \approx \rho_4$

\subsubsection{Test coefficient across regression}

Note: do not read off regression table for this

\begin{lstlisting}
reg Y x1
est store reg1
reg Y x1 x2
est store reg2
suest reg1 reg2
test [reg1_mean]D = [reg2_mean]D
\end{lstlisting}

\section{Instrument Variable and 2SLS}

IV enable causal effect estimates with omitted variable

Valid IV satisfies two conditions
\begin{itemize}
    \item Relevance condition: $Cov(s_i, Z_i) \neq 0$
    \item Exclusion restriction: $Cov(\eta_i, Z_i) = 0$
\end{itemize}
Where
\begin{tabulary}{\linewidth}{l @{ := } L}
    $s_i$ & treatment\\
    $Z_i$ & IV\\
    $\eta_i$ & error in short regression
\end{tabulary}

\subsection{IV and Causality}

Assume $s_i \perp \nu_i | A_i$ (CIA), $A_i$ not observed
\begin{align*}
    \text{(short) } Y_i &= \alpha + \rho s_i + \eta_i\\
    \text{(long) } Y_i &= \alpha + \rho s + \gamma A_i + \nu_i\\
    \Rightarrow \eta_i &= \gamma A_i + \nu_i \text{ (OVB)}
\end{align*}

Omitted variable bias:\\
Comparing OLS and IV estimator gives sign of bias
\begin{align*}
    \rho^{OLS} &= \frac{Cov(Y_i, s_i)}{Var(s_i)}\\
               &=\frac{Cov(\alpha + \rho s + \gamma A_i + \nu_i, s_i)}{Var(s_i)}\\
               &=\rho + \gamma \frac{Cov(A_i, s_i)}{Var(s_i)}
\end{align*}

\subsubsection{Exclusion restriction}
\begin{itemize}
    \item Instrument is independent of potential outcomes (condition on covariates)
    \item $Z$ has no effect on outcomes other than through $S$
\end{itemize}

\subsection{Two-Stage Least Squares (2SLS)}
2SLS allows adding covariates (controls) and combine multiple instruments

First stage
\begin{align*}
    s_i = \pi_{11}Z_i + X_i'\pi_{12} + \xi_{1i}
\end{align*}
Reduced Form
\begin{align*}
    Y_i = \pi_{21}Z_i + X_i'\pi_{22} + \xi_{2i}
\end{align*}
Second stage (modified short regression)
\begin{align*}
    Y_i = \rho \hat{s}_i + X'_i\theta + (\eta_i + \rho \xi_{1i})
\end{align*}
$\rho$ is determined by
\begin{align*}
    \rho = \frac{\pi_{21}}{\pi_{11}}
\end{align*}

\subsubsection{Estimate causal effect $\rho$}
Causal effect is ratio of reduced form / first stage:
\begin{align*}
    \rho &= \frac{Cov(Y_i, Z_i)}{Cov(s_i, Z_i)}\\
         &= \frac{Cov(Y_i, Z_i)/V(Z_i)}{Cov(s_i, Z_i)/V(Z_i)}
\end{align*}

Arise from
\begin{align*}
    \frac{Cov(Y_i, Z_i)}{Cov(s_i, Z_i)} &= \frac{Cov(\alpha+\rho s + \gamma A_i + \nu_i, Z_i)}{Cov(s_i, Z_i)}\\
                                        &=\rho \frac{Cov(s_i, Z_i)}{Cov(s_i, Z_i)} = \rho
\end{align*}

\subsubsection{Wald Estimator}
Suppose IV $Z_i$ is dummy variable
\begin{align*}
    \rho = \frac{E(Y_i|Z_i=1)-E(Y_i|Z_i=0)}{E(S_i|Z_i=1)-E(S_i|Z_i=0)}
\end{align*}

\subsection{2SLS bias}
Note:
\begin{itemize}
    \item 2SLS estimator is consistent\\
        Large sample: $E(\hat\rho)\approx\rho$
    \item 2SLS estimator is biased\\
        Finite sample: $E(\hat\rho)\neq\rho$
    \item When first stage $F-$stat = 0 $\Rightarrow$ 2SLS = OLS
    \item Bias vanish when $F-$stat in first stage is large $(>10)$
    \item Useless IV increase bias \\(var with no effect on first-stage $R^2$)
\end{itemize}

\subsection{Identify IV}

Good instruments come from institutional knowledge and idea about
process determining the variable of interest

\subsubsection{Test relevance condition $\Rightarrow Cov(s_i, Z_i)\neq0$}

$F$ test $>10$ for all IVs in the first stage regression 

\subsubsection{Test exclusion restriction $\Rightarrow Cov(\eta_i, Z_i)=0$}

Difficult as $\eta_i$ not observed

\begin{enumerate}
    \item Over identifying restriction test\\
        If $Q>K$ (num of IV $>$ num of treatment var)\\
        and IV are chosen with the same logic
        \begin{align*}
            H_0 &: \text{instruments are all valid}\\
            H_1 &: \text{at least one of the IV are not valid}
        \end{align*}
    \item Qualitative argument\\
        Argue that $Z, Y$ not related
    \item Falsification test\\
        Find another non-treatment sample s.t. $Y_i = \beta Z_i + \epsilon_i \Rightarrow \beta=0$
\end{enumerate}

Note: $Q<K \Rightarrow$ need more IV. $Q=K\Rightarrow$ just enough IV

\section{Panel Data and Fixed Effects}

Fixed Effect model
\begin{align*}
    Y_{it} = \alpha_i + \lambda_t + \rho D_{it} + X_{it}'\delta + \epsilon_{it}
\end{align*}

Estimated using: OLS + dummies, de-mean, first difference

\subsection{Cross Section/Time Series/Panel Data}

\begin{tabulary}{\linewidth}{l @{ : } L}
    Cross section & many subjects at the same point of time\\
    Time series & sequence of data points over a time interval\\
    Panel data & behavior of entities are observed across time (aka longitudinal data)\\
    Balanced panel & all data across the years are observed\\
    Unbalanced panel & missing data for some years
\end{tabulary}

\subsection{Solving OVB}
To solve omitted variable bias in cross-sectional data
\begin{itemize}
    \item Find a proxy
    \item Find a valid IV
\end{itemize}

Panel data further use
\begin{itemize}
    \item Random effect (RE) models\\
        (Assumes no fixed effect)
    \item Fixed effect method \\
        (Eliminate time-invariant individual characteristics)
    \item IV, Differences in Differences
\end{itemize}

\subsection{Fixed Effect Model: Causal Inference}
treatment ($D_{it}$), treatment effect ($\rho$)

regression model ($A_i$ not observed)
\begin{align*}
    Y_{it} &= \alpha + \lambda_t + \rho D_{it} + X_{it}' \delta + A_i'\gamma + \epsilon_{it}\\
    u_{it} &:= A_i'\gamma + \epsilon_{it} \text{ (OVB)}
\end{align*}

fixed effect model (solve OVB by absorb $A_i$)
\begin{align*}
    Y_{it} &= \alpha_i + \lambda_t + \rho D_{it} + X_{it}'\delta + \epsilon_{it}\\
    \alpha_i &= \alpha + A'_i\gamma \text{ (individual fixed effect)}
\end{align*}

\subsection{FE: Estimation}
OLS regression with dummies
\begin{align*}
    Y_{it} &= \lambda_t + \rho D_{it} + X_{it}'\delta + \sum_{i=1}^{N-1}\alpha_i I_i + \epsilon_{it}
\end{align*}
Within estimator (de-mean)
\begin{align*}
    Y_{it} - \bar Y_i &= (\lambda_t - \bar\lambda) + \rho (D_{it}
    -\bar D_i) + (X_{it} - \bar X_i)'\delta + (\epsilon_{it} -
    \bar\epsilon_{i})\\ \bar{Y}_{i} &= \alpha_i + \bar\lambda
    +\rho\bar D_i + \bar{X}_i'\delta + \bar\epsilon_{i}
\end{align*}
First differencing
\begin{align*}
    \Delta Y_{it} &= \Delta \lambda_t + \rho \Delta D_{it} + \Delta
    X'_{it}\delta + \Delta \epsilon_{it}
\end{align*}

Remarks
\begin{itemize}
    \item With 2 periods, all 3 methods are algebraically the same
    \item First differencing introduces serial correlation of error terms (not recommended)
    \item Interpretation (all methods):\\
        for a given individual/firm/country, as $X$ varies across time by one unit, $Y$
        increases or decrease by $\rho$ units
\end{itemize}

\subsubsection{FE vs OLS}

\begin{itemize}
    \item Compare OLS vs FE gives sign of selection bias
    \item pooled regression: OLS estimate without fixed effects
    \item we require variations in FE: cannot investigate
        time-invariant variable (e.g. union status unchanged across
        time)
    \item FE controls for all time-invariant differences between individuals
\end{itemize}

\section{Differences-In-Differences}


\begin{center}
\begin{tabular}{|l | l | l|}
    \hline
    & Treatment & Control \\
    \hline
    Pre-Program & \makecell{A \\ $\bar{Y}_{Pre}^{Treatment}$} &
    \makecell{B \\ $\bar{Y}_{Pre}^{Control}$} \\
    \hline
    Post-Program & \makecell{C \\ $\bar{Y}_{Post}^{Treatment}$} &
    \makecell{D \\ $\bar{Y}_{Post}^{Control}$} \\
    \hline
\end{tabular}
\end{center}
DID estimator
\begin{align*}
    \left(\bar{Y}_{Post}^{Treatment} - \bar{Y}_{Pre}^{Treatment}\right) &- \left( \bar{Y}_{Post}^{Control} -
    \bar{Y}_{Pre}^{Control}  \right)\\
        (C - A) &- (D - B)
\end{align*}

\begin{tabular}{l @{ := } l}
    Selection bias & $A-B$\\
    Time trend & $D-B$\\
    Treatment effect & $(C-A)-(D-B)$ \\&$ (C-D) - (A-B)$
\end{tabular}

\subsection{DID Regression}

\begin{align*}
    Y_{it} &= \alpha + \gamma D_i + \eta Post_t + \beta D_i \cdot Post_t + \epsilon_{it}
\end{align*}

\begin{tabular}{l @{ := } l}
    $D_i$ & indicator for observation is treatment group\\
    $Post_t$ & indicator for time is after treatment\\
    $\beta$ & treatment effect (DID estimator)\\
    $\alpha$ & pre-program mean in control group\\
    $\gamma$ & selection bias\\
    $\eta$ & time trend
\end{tabular}

\subsection{DID Assumption}


\subsubsection{Parallel (common) Trends}

Test $\beta_{\tau}=0$ for $\tau <0$

\begin{align*}
    Y_{ist} = \alpha + \gamma D_i + \eta Post_t + \sum_{\tau=-m, \tau
    \neq -1}^{q} \beta_{\tau} W_t^{\tau} \cdot D_i + \epsilon_{ist}
\end{align*}

\begin{tabulary}{\linewidth}{l @{ := } L}
    $W_t^\tau$ & indicator of time is $\tau$ periods ago\\
               & Pre-period: $\tau < 0$, Post-Period: $\tau \geq 0$\\
    $\tau \neq -1$ & base group, period right before treatment\\
    $\beta_\tau$ & average $(Y-\gamma)$ difference between treatment
    and control group at $t=\tau$ (similar to dynamic effect model)
\end{tabulary}

Common trends assumption:
\begin{itemize}
    \item Selection bias relates to fixed characteristics of individuals
        \subitem[$\Rightarrow$] selection bias magnitude does not change over time
    \item Time trend is the same for treatment and control groups
\end{itemize}

Note:

Ideally test if untreated outcome of treatment and control are parallel in post-treatment.
But counterfactual is not observed. Therefore, test for parallel trend before treatment.


\subsubsection{No Omitted Variables that Correlate with Treatment Status}

Estimation is biased if there are other factors affecting the
difference in trends between treatment and control.

Difficult to test, recommend:
\begin{itemize}
    \item Placebo (Falsification) test
        \subitem[1.1] Exploit a population not affected by policy
        \subitem[1.2] Use outcome variable not affected by policy, but
        affected by potential unobserved policy shocks
    \item IV strategy
        \subitem exogenous variation of treatment status
\end{itemize}

\subsubsection{Endogenous Intervention}

DID estimation is appropriate when interventions are as good as random, conditional on the controls

\begin{itemize}
    \item treatment is randomised
    \item if possible endogeneity occur (treatment group is selected),
        DID is biased
    \item idea: the post treatment trend is expected to change even
        without policy (bias upwards)
\end{itemize}



\subsection{Continuous treatment intensity}

\begin{align*}
    Y_{it} &= \alpha + \delta S_i \cdot Post_t + \beta S_i + \eta Post_t + \epsilon_{it}
\end{align*}

$S :=$ continuous variable, measuring treatment intensity.\\
$\delta :=$ measures treatment effect for the continuous treatment.

\subsubsection{Adding fixed effects}

\begin{align*}
    Y_{it} &= \alpha + \delta S_i \cdot Post_t + \theta_t + \delta_i + \epsilon_{it}
\end{align*}

$S :=$ continuous variable, measuring treatment intensity.\\
$\delta :=$ measures treatment effect for the continuous treatment.\\
$\theta_t :=$ time fixed effects\\
$\delta_i :=$ individual fixed effects

\subsection{DID with IVs}

If $S$ is endogenous (treatment group is associated with lower
outcome), we can find an instrument for $S$

First-Stage DID:
\begin{align*}
    S_i \cdot Post_t &= \alpha + \delta Z_i \cdot Post_t + \theta_t +
    \delta_i + \epsilon_{it}
\end{align*}

Reduced Form DID:
\begin{align*}
    Y_{it} &= \alpha + \delta^r Z_i \cdot Post_t + \theta_t + \delta_i
    + \epsilon_{it}
\end{align*}

Second stage:
\begin{align*}
Y_{it} &= \alpha + \gamma \hat{S_i}\cdot \hat{Post_t} + \theta_t +
\delta_i + \epsilon_{it}
\end{align*}

$S$ can be continuous or discrete\\
$\gamma = \delta^r/\delta$ measures the causal effect

\subsection{Clustered Standard Errors}

\begin{itemize}
    \item Panel data introduce serially correlated error problem
        within the cross section across years.
    \item Ignoring serial correlation underestimate the standard
        error. (exaggerate precision of regression estimates)
\end{itemize}

\subsubsection{Bias in un-clustered standard errors}

Bias: overly rejected $H_0: \beta = 0$ ($67.5\%$ instead of $5\%$)

\subsubsection{Solution1: Ignore time series data}

Aggregating data into one pre and one post period

\subsubsection{Solution2: Clustered standard error}

Cluster standard error at state, year, state x year level

\begin{itemize}
    \item Require clusters to be sampled randomly
    \item Require large clusters ($>10$)
\end{itemize}

\section{Regression Discontinuity}

Identification assumption: all factors (other than assignment) are
evolving ``smoothly'' with respect to X $\Leftrightarrow
E[Y_{0i}|X_i]$ and $E[Y_{1i}|X_i]$ are continuous in $X_i$ at $c$.

\subsection{Sharp RD}

Treatment is deterministic function of assignment D

\begin{align*}
    D &= \begin{cases}
        1, & X\geq c\\
        0, & X < c
    \end{cases}\\
    Y &= \alpha + \tau D + f(X) + \epsilon
\end{align*}
$\tau :=$ treatment effect

\subsection{RD Regression}

\begin{align*}
    E[Y_i | X_i] &= E[Y_{0i} | X_i] + (E[Y_{1i}|X_i] - E[Y_{0i}|X_i])D_i\\
    \tau &:= E[Y_{1i}|X_i] - E[Y_{0i}|X_i]\\
    E[Y_{0i}|X_i] &:= \alpha + f(X_i)\\
    \Rightarrow Y_i &= \alpha + f(X_i) + \tau D_i + \epsilon_i
\end{align*}

$D_i = I(X \geq c)$, $c$ = cutoff

\subsubsection{Polynomial Method}
Approximate $f(X) = \sum_{p=1}^P \beta_p X^p$, polynomial terms

$\tilde{X_i} = X_i - c$
\begin{align*}
    Y_i &= \alpha + \beta_{01} \tilde{X_i} + \beta_{02} \tilde{X_i^2}
    + \cdots + \beta_{0p}\tilde{X_i^p}\\ &+ \tau D_i + \beta_{11}^*
    D_i \tilde{X_i} + \beta_{12}^* D_i \tilde{X_i^2} + \cdots +
    \beta_{1p}^* D_i \tilde{X_i^p} + \eta_i
\end{align*}
\begin{tabulary}{\linewidth}{l @{ := } L}
    $\beta_{0p}$ & correlation between $Y, X$ of the control group\\
    $\beta_{1p}^*$ & $\beta_{1p} - \beta_{0p}$\\ & incremental
    correlation between $Y, X$ relative to control group\\ $\tau $ &
    treatment effect
\end{tabulary}

\subsubsection{Local Linear Regression}
Select only the data certain width $h$ around cutoff $c$

$c-h \leq X \leq c+h$, $\tilde{X_i} = X_i - c$
\begin{align*}
    Y_i &= \alpha + \beta_0 \tilde{X_i} + \tau D_i + \beta_1^* D_i
    \tilde{X_i} + \eta_i
\end{align*}
\begin{tabulary}{\linewidth}{l @{ := } L}
    $\beta_0$ & linear correlation between $Y, X$ of control group\\
    $\beta_1^*$ & $\beta_1 - \beta_0$, incremental linear correlation
    between $Y, X$ of treatment relative to control
\end{tabulary}

Note:
\begin{itemize}
    \item Can be combined with polynomial method
    \item Larger bandwidth: more precise (lower variance) but higher
        bias
\end{itemize}

\subsection{Model selection}

\subsubsection{AIC}

Choosing order of polynomial $p$
\begin{align*}
    AIC(p) &= N ln(\hat\sigma^2(p)) + 2p\\
    p_{AIC}^{opt} &= \arg \min_p AIC(p)
\end{align*}

$\hat\sigma^2(p)$ := mean squared error of regression\\
$p$ := num of regressors

\subsubsection{Bin Dummies}

Choosing order of polynomial $p$

Test $H_0: \gamma_k = 0$ with $F-$test\\
increase polynomial order until $\gamma_k=0$
\begin{align*}
    Y_i &= \alpha + \beta_0 \tilde{X_i} + \tau D_i + \beta_1 D_i
    \tilde{X_i} + \sum_{k=1}^b \gamma_k \cdot I_k + \eta_i
\end{align*}
\begin{tabulary}{\linewidth}{l @{ := } L}
    $I_k$ & $I(X \in {bin}_k)$\\
    ${bin}_k$ & $[l, l+2h), [l+2h, l+4h), \cdots, [u-2h, u]$\\
    $l$ & lowest $X$\\
    $u$ & highest $X$\\
    bin width & $2h$\\
    num of bin & $(u-l)/2h$
\end{tabulary}

\subsubsection{Cross-validation Procedure}
Only method for both bandwidth choice $h$ and polynomial $p$
\begin{align*}
    CV(h) &= \frac{1}{N} \sum_{i=1}^N (Y_i - \hat Y)^2\\
    h_{CV}^{opt} &= \arg \min_h CV(h)
\end{align*}
Select only data within $c\pm h$ and conduct train/test split\\
Testing assumption horizontal line fit is suitable

\subsection{Graphical Analysis}

Divide assignment variable into a number of bins and plot the average
outcome value against mid-points of the bins

\subsection{Valid or Invalid RD: Sorting}

Individuals influence the assignment variable.
For example, individuals check their answers to avoid failing

\begin{tabular}{| l | l | l|}
    \hline
    & Marginal pass & Marginal fail\\
    \hline
    Case I & Type A and B & Only Type B\\
    Case II & Type A and B & Type A and B\\
    \hline
\end{tabular}

Case I: invalid RD\\
Case II: valid RD, individual has imprecise control

\subsection{Testing Validity}

\begin{itemize}
    \item Test I: test whether the covariates $W$ are balanced at the
        threshold.
        \subitem e.g. income, age, and observed characteristics not
        affected by treatment
        \subitem However, impossible to test unobserved
        characteristics
    \item Test II: test if density of $X$ (assignment variable) is
        continuous
        \subitem  jump in density indicate sorting
\end{itemize}

\subsubsection{Test I: no Discontinuity in covariate}

$x-$axis: assignment variable\\
$y-$axis: pre-determined variable
\begin{align*}
    W &= a + bD + f(X) + \epsilon
\end{align*}
$b=0\Rightarrow$ pre-determined characteristics is balanced

\subsubsection{Test II: no Discontinuity in assignment var density}
$x-$axis: assignment variable\\
$y-$axis: density of assignment variable
\begin{align*}
    Density(X) &= a + bD + f(X) + \epsilon
\end{align*}
$b=0\Rightarrow$ density of assignment variable is continuous

\subsection{Fuzzy RD}

Exploits Discontinuity in probability of treatment conditional on assignment D
\begin{align*}
    P(D_i = 1 | X_i) &= \begin{cases}
        g_1(X_i), & X_i > c\\
        g_0(X_i), & X_i \leq c
    \end{cases}\\
        g_0(X_i) &\neq g_1(X_i)\\
        P(D_i = 1 | X_i) &= g_0(X_i) + [g_1(X_i) - g_0(X_i)]T_i\\
        T_i &= I(X_i \geq c)
\end{align*}

The discontinuity $T_i$ becomes an instrument variable for treatment status $D$

\subsubsection{Imperfect Compliance: Fuzzy RD Design}

First-stage
\begin{align*}
    D &= \gamma + \delta T + g(X) + \nu
\end{align*}

Reduced Form
\begin{align*}
    Y &= \alpha_r + \tau_r T + f_r(X) + \epsilon_r
\end{align*}

Second stage
\begin{align*}
    Y &= \alpha + \tau \hat{D} + f(X) + (\epsilon + \tau\nu)\\
\end{align*}
\begin{tabulary}{\linewidth}{l @{ := } l @{ } L}
    $T$ & $I(X_i\geq c)$& instrument for $D$\\
    $\tau_r$ & $\tau\delta$& Intent-to-treat (ITT) effect\\
    $\tau$ & $\frac{\tau_r}{\delta}$& treatment effect
\end{tabulary}

\subsubsection{Estimation}

$\tilde{X_i} = X_i - c$

First-stage
\begin{align*}
        D_i &= \gamma_{00} + \gamma_{01}\tilde{X_i} +
        \gamma_{02}\tilde{X_i^2} + \cdots + \gamma_{0p}\tilde{X_i^p}\\
            &+ \pi T_i + \gamma_{11}^* T_i \tilde{X_i} + \gamma_{12}^*
            T_i \tilde{X_i^2} + \cdots + \gamma_{1p}^* T_i
            \tilde{X_i}^p + \epsilon_i
\end{align*}

Second-stage
\begin{align*}
    Y_i &= \alpha + \beta_{01}\tilde{X_i} + \beta_{02}\tilde{X_i}^2 +
    \cdots + \beta_{0p}\tilde{X_i}^p \\ &+ \tau\hat{D_i} +
    \beta^*_{11}\hat{D_i}{\tilde{X_i}} +
    \beta_{12}^*\hat{D_i}{\tilde{X_i}^2} + \cdots + \beta_{1p}^*
    \hat{D_i}{\tilde{X_i}^p} + \eta_i
\end{align*}
$T_i, T_i\tilde{X_i}, \cdots, T_i\tilde{X_i^p}$ are instruments for
$D_i, D_i\tilde{X_i}, \cdots, D_i\tilde{X_i^p}$

\subsubsection{Assumption}

Same assumption as standard IV framework (LATE)
\begin{itemize}
    \item Monotonicity: assignment $X$ must result in same direction
        on outcome
    \item Excludability: assignment $X$ cannot impact outcome expect
        through receipt of treatment
\end{itemize}

\end{multicols}
\end{document}
