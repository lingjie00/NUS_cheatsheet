\documentclass[12pt,landscape]{article}
\usepackage{multicol}
\usepackage{calc}
\usepackage{ifthen}
\usepackage[landscape]{geometry}
\usepackage{amsmath,amsthm,amsfonts,amssymb}
\usepackage{color,graphicx,overpic}
\usepackage{hyperref}
\usepackage{soul} %for highlight
\usepackage{xcolor} %color definition
\usepackage{sectsty} %change section color


\pdfinfo{
  /Title (EC3101 Microeconomics cheatsheet)
  /Creator (Lingjie)}

% This sets page margins to .5 inch if using letter paper, and to 1cm
% if using A4 paper. (This probably isn't strictly necessary.)
% If using another size paper, use default 1cm margins.
\ifthenelse{\lengthtest { \paperwidth = 11in}}
    { \geometry{top=.5in,left=.5in,right=.5in,bottom=.5in} }
    {\ifthenelse{ \lengthtest{ \paperwidth = 297mm}}
        {\geometry{top=1cm,left=1cm,right=1cm,bottom=1cm} }
        {\geometry{top=1cm,left=1cm,right=1cm,bottom=1cm} }
    }

% Turn off header and footer
\pagestyle{empty}

% Redefine section commands to use less space
\makeatletter
\renewcommand{\section}{\@startsection{section}{1}{0mm}%
                                {-1ex plus -.5ex minus -.2ex}%
                                {0.5ex plus .2ex}%x
                                {\normalfont\large\bfseries\color{red}}}
\renewcommand{\subsection}{\@startsection{subsection}{2}{0mm}%
                                {-1explus -.5ex minus -.2ex}%
                                {0.5ex plus .2ex}%
                                {\normalfont\normalsize\bfseries\color{blue}}}
\renewcommand{\subsubsection}{\@startsection{subsubsection}{3}{0mm}%
                                {-1ex plus -.5ex minus -.2ex}%
                                {1ex plus .2ex}%
                                {\normalfont\small\bfseries\color{violet}}}
\makeatother

% Define BibTeX command
\def\BibTeX{{\rm B\kern-.05em{\sc i\kern-.025em b}\kern-.08em
    T\kern-.1667em\lower.7ex\hbox{E}\kern-.125emX}}

% Don't print section numbers
\setcounter{secnumdepth}{0}


\setlength{\parindent}{0pt}
\setlength{\parskip}{0pt plus 0.5ex}

%My Environments
\newtheorem{example}[section]{Example}
% -----------------------------------------------------------------------

\begin{document}
\raggedright
\footnotesize
\begin{multicols}{3}


% multicol parameters
% These lengths are set only within the two main columns
%\setlength{\columnseprule}{0.25pt}
\setlength{\premulticols}{1pt}
\setlength{\postmulticols}{1pt}
\setlength{\multicolsep}{1pt}
\setlength{\columnsep}{2pt}

\begin{center}
     \Large{\underline{EC3101 Microeconomics II}} \\
     {Lingjie, \today}
\end{center}

\section{Inter-temporal Choice}
	Future value, FV = $m(1+i)$\\
	Present value, PV = $\frac{FV}{(1+i)}$

\subsubsection{Inter-temporal Utility}
	$u(C_1, C_2) = C_1 + \delta C_2$, $\delta$ is discount rate

\subsubsection{Inter-temporal Budget Constraint}
	$(C_1,C_2) = (m_1,m_2) \Rightarrow$ consumer don't save/borrow\\
	$C_1 + \frac{C_2}{1+i} = m_1 + \frac{m_2}{1+i}$, $m_i$ in units of goods\\

\subsection{Changes in Nominal Interest Rate}
	$MRS_{C_1, C_2} = 1+i$\\
	changes in $r$ affects slope of curve, higher steeper\\
	\begin{tabular}{lccc}
							&		saver			&		borrower				& neither\\
		\hline
		$\uparrow i$ \vline		&		$\uparrow U$		&		$\downarrow U$		& unchanged\\
		\hline
		$\downarrow i$ \vline	&		$\downarrow U$	&		$\uparrow U$			& unchanged\\
		\hline
	\end{tabular}\\
	Neither result in unchanged utility if and only if income adjust accordingly\\
	With increased $i$, borrower might become saver\\

\subsection{Changes in Price}
	If $p \neq 1$: $p_1C_1+\frac{p_2}{1+r}C_2 = m^*_1 + \frac{m^*_2}{1+i}$, $m^*_i$ = $m_i$ money\\
	new MRS $= (1+i)\frac{p_1}{p_2}$\\
	\begin{tabular}{lcc}
							&		$C_1$			&		$C_2$				\\
		\hline
		$\uparrow p_1$ \vline	&		$\downarrow$		&		$\uparrow$			\\
		$\downarrow p_1$ \vline	&		$\uparrow$		&		$\downarrow$			\\
		\hline
		$\uparrow p_2$ \vline	&		$\uparrow$		&		$\downarrow$			\\
		$\downarrow p_2$ \vline	&		$\downarrow$		&		$\uparrow$			\\
		\hline
	\end{tabular}\\

\subsection{Changes in inflation}
	$p_1(1+\pi) = p_2$ or $\pi = \frac{p_2-p_1}{p_1}$\\
	If $\pi \neq 0, p_1 = 1$: $c_1 + \frac{1+\pi}{1+i}c_2 = m^*_1 + \frac{m^*_2}{1+i}$
	\begin{tabular}{lccc}
							&		saver			&		borrower				& neither\\
		\hline
		$\uparrow \pi$ \vline		&		$\downarrow U$		&		$\uparrow U$		& unchanged\\
		\hline
		$\downarrow \pi$ \vline	&		$\uparrow U$	&		$\downarrow U$			& unchanged\\
		\hline
	\end{tabular}\\
	Neither result in unchanged utility if and only if income adjust accordingly\\

\subsubsection{Real Interest Rate}
	$1+\rho = \frac{1+i}{1+\pi}$, $\rho$ is real interset rate\\
	$\rho = \frac{i-\pi}{1+\pi} \approx i-\pi$ for small $\pi$

\section{Uncertainty}
	Expected Utility: $u(c_1, c_2, \pi_1, \pi_2) = \pi_1\nu(c_1) + \pi_2\nu(c_2)$\\
	Expected Money: $\pi_1c_1 + \pi_1c_1$\\
	Given concave utility $\Rightarrow \frac{\partial U}{\partial c} = MU_c(c) < 0$, $\uparrow c \downarrow U$

\subsection{Risk Measurement}
	\begin{tabular}{l@{ : }l}
		Risk aversion			& EU $<$ U(EM)\\
		Certainty equivalence	& $U^{-1}(EU)$
	\end{tabular}
	
\subsubsection{Risk Premium}
	RP = EM - CE\\
	RP is 'willingness-to-pay' to avoid risk, higher more risk averse\\
	CE is the certainty amount needed to achieve the utility under risk

\subsubsection{Arrow-Pratt}
	$-\frac{u''(w)}{u'(w)}$, higher more risk averse

\subsection{State-Contingent Budget Constraints}
	\textit{States of Nature}: outcomes of random events\\
	\textit{Contingent consumption plan}: specification of amount consumed in different state of nature\\
	\textit{Insurance premium}: price of insurance, $\gamma$\\
	\begin{tabular}{ll}
		w/o insurance			&	$C_{-} = m-L, C_{+} = m$\\
		w insurance			&	$C_{-} = m-L+(1-\gamma) K$\\
							&	$C_{+} = m-\gamma K$\\
		budget constraint		&	$C_{+} = -\frac{\gamma}{1-\gamma}C_{-} + \frac{m-\gamma L}{1-\gamma}$\\
		$MRS_{c_{-}, c_{+}}$	&	$\frac{\pi_{-}MU(C_{-})}{\pi_{+}MU(C_{+})} = \frac{\gamma}{1-\gamma}$, price ratio
	\end{tabular}

\subsection{Market for Insurance}
	$\pi_{\text{insurance}} = \gamma K - (\pi_{-}K + \pi_{+}0) = (\gamma - \pi_{-})K$\\
	Buy insurance when: $U(C_{+}-\gamma K) \geq EU$

\subsection{Insurance pricing}
	\begin{tabular}{ll}
		competitive pricing		&		$\pi_{\text{insurance}} =0\Rightarrow \gamma = \pi_{-}$\\
		fully insurance			&		$\frac{MU(C_{-})}{MU(C_{+})}=1 = \frac{C_-}{C_+}$\\
		\hline
		unfair pricing			&		$\pi_{\text{insurance}} >0\Rightarrow \gamma > \pi_{-}$\\
		partial insurance		&		$\frac{MU(C_-)}{MU(C_+)} > 1 \Rightarrow \frac{C_-}{C_+} < 1$\\
		\hline
		Full insurance 			& 		$C_- = C_+$, $45^o$ line\\
		Partial insurance		&		$C_- < C_+$, nearer to $C_+$
	\end{tabular}
	
	

\section{Monopoly}
	Single seller for the market with downward sloping market demand\\
	Producer surplus $PS$ below price above supply\\
	Consumer surplus $CS$ above price below demand
	
\subsection{Profit Maximisation}
		$\max_q \pi = p(q)q - c(q)$ till $MR =MC$

\subsection{Price Elasticity of Demand, PED}
	PED: $\epsilon = \left[\frac{dq}{q}/\frac{dp}{p}\right] \leq 0, dq=1$\\
	$p'(q)q + p(q) = c'(q)=p(q)\left[1+\frac{dp}{p}\frac{q}{dq}\right] = p(q)[1-\frac{1}{|\epsilon|}] $\\
	In demand curve, $\epsilon$ decrease from elastic to inelastic

\subsubsection{PED cases ($\epsilon \leq 1$ and Markup Pricing)}
	\begin{tabular}{l@{ : }ll}
		$\epsilon = \infty$	&	$p(q) = c'(q)$ 							& 	perfect competition\\
		$\epsilon \leq 1$	&	$q \downarrow 0=0$						&	near 0 production\\
		$\epsilon > 1$		&	$p(q) = \frac{c'(q)}{[1-\frac{1}{|\epsilon|}]}$		&	markup prices
	\end{tabular}

\subsection{Inefficiency of Monopoly}
	Since $Q^* < Q_{m}$, monopoly produce less than market equilibrium, $CS^* < CS_{m}$ and DWL exist

\subsection{First-degree Price Discrimination}
	$P=$ demand, PS = market surplus

\subsection{Third-degree Price Discrimination}
	Assumptions
	\begin{itemize}
		\item Perfect identification of different markets
		\item No arbitrage across groups
	\end{itemize}
	solve $\max_{q_1, q_2} p_1(q_1)q_1 + p_2(q_2) - c_1(q_1) - c_2(q_2)$

\subsubsection{Uniform Pricing}
	Firm charging the same price for both markets
	$p_1(q) > p_2(q)$
	\begin{align*}
	Q &= p_1^{-1}1_{\{P> p_2(0)\}} + [p_1^{-1}+p_2^{-1}]1_{\{P\leq p_2(0)\}}\\
	P &= p_11_{\{Q>Q(p_2(0))\}} + p(p_1^{-1}+p_2^{-1})1_{\{Q\leq Q(p_2(0))\}}
	\end{align*}
	Solve for MR = MC and decide on Q, P, $\pi$

\subsection{Barriers to Entry}
	Types: legal, strategic, structural

\subsection{Natural Monopoly}
	Firm with Economies of scale and low demand (relative to cost) $\Rightarrow$ downward sloping $ATC > MC$

\subsection{Regulating Natural Monopoly}
	\begin{tabular}{ll}
		MC pricing		&		$MC=P<ATC \Rightarrow \pi < 0$\\
		ATC pricing		&		$MC<P=ATC \Rightarrow \pi =0$\\
		Qty tax,$\tau$		&		$\uparrow P, \downarrow Q$, distort market
	\end{tabular}
	However, ATC pricing incur high regulatory cost and no firms have no incentive to be cost efficient

\subsubsection{Quantity Tax Levied on Monopolist}
	\begin{tabular}{l@{ : }l}
		w/o $\tau$		&	$p(q)\left[1+\frac{1}{\epsilon}\right] = k \Rightarrow p(q) = \frac{\epsilon k}{1+\epsilon}$\\
		w $\tau$		&	$\frac{\epsilon(k+\tau)}{1+\epsilon} \Rightarrow \frac{\epsilon k}{1+\epsilon} + \frac{\epsilon \tau}{1+\epsilon}$
	\end{tabular}\\
	Since monopolist only operates when $\epsilon < 1$, $\frac{\epsilon \tau}{1+\epsilon} > 1$, more elastic demand means firms pass even more than full time to consumer.\\
	However, in reality firms do not pass more than tax to consumer. This result also assume whole demand curve to be constant elasticity (not true)

\section{Oligopoly}

\subsection{Cournot}
	Assumptions:
	\begin{itemize}
		\item identical product
		\item independent and simultaneously strategy
		\item firms choose output level
	\end{itemize}
	Payoffs and best response function:
	\begin{align*}
		&\max_{q_i} P(q_1+q_2)q_i -C_i(q_i)~\forall~i\in N
	\end{align*}

\subsection{Stackelberg}
	Assumptions:
	\begin{itemize}
		\item Stackelberg leader set quantity independently
		\item follower observe leader and set quantity
	\end{itemize}
	1. Solve followers' best response per \textit{Cournot}\\
	2. Solve leaders' best response function
	\begin{align*}
		\max_{q_i} P(q_i+q_j(q_i))q_i - C_i(q_i)~\forall~i,j\in N
	\end{align*}

\subsection{Collusion}
	Firms collude to form cartel and choose $Q_{mkt}$
	\begin{align*}
		\max_Q P(Q)Q - C(Q)
	\end{align*}
	Decision to produce
	\begin{tabular}{ll}
		$C_i(Q) < C_j(Q)$		&		only firm $j$ will produce\\
		$C_i(Q) = C_j(Q)$		&		firms share $Q$ till $MC_i = MC_j$
	\end{tabular}\\
	Collusion is not sustainable in static game\\
	In dynamic game, firms punish by 'grim trigger strategy', not cooperate again\\
	$\alpha$ = payoff in collusion\\
	$\beta$ = payoff in Cournot\\
	$\gamma$ = payoff in Cournot with opp in collusion\\
	$\delta$ = discount factor
	\begin{align*}
		\pi_{coop} &= \pi_{cheat}\\
		\alpha + \delta\alpha + \delta^2\alpha + \cdots &= \gamma + \delta\beta + \delta^2\beta + \cdots\\
		\sum_{i=[0,\infty]}{\delta^i\alpha} &= \gamma + \sum_{i=[1,\infty]}{\delta^i\beta}\\
		\frac{\alpha}{1+\delta} &= \frac{\gamma}{1+\delta}
	\end{align*}
	
	
\subsection{Bertrand}
	Assumptions:
	\begin{itemize}
		\item identical product
		\item constant $MC=c$, no fixed cost
		\item firms choose price level
	\end{itemize}
	Cases:\\
	\begin{tabular}{ll}
		$P_i > P_j$			&		$Q_i =0, P_i = P_j - \delta \downarrow_0$\\
		$P_i = P_j$			&		$Q_i = Q_j$\\
		$P_j > P_{monopoly}$	&		$P_i = P_{monopoly}$\\
		$P_j < MC$			&		$P_i = MC$
	\end{tabular}

\subsubsection{Bertrand, Cournot, Stackelberg, Collusion}
	\begin{tabular}{lcccc}
				&	Bertrand	&	Cournot	&	Stackelberg	&	Collusion\\
		$P^*$	&	1		&	2		&	3			&	4\\
		$Q^*$	&	4		&	3		&	2			&	1\\
		$\pi$		&	0		&	1		&	2			&	3
	\end{tabular}
	\centerline{best for consumers $\longrightarrow$ best for firms}

\subsubsection{Capacity Constraint}
	In this mod we assume single firm can satisfy the entire market


\end{multicols}
\end{document}